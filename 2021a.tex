% このファイルは日本語用です。
% 次の行は変更しないでください。
\documentclass[ja]{2021a}
%%%%%%%%%%%%%%%%%%%%%%%%%%%%%%%%%%%%%%%%%%%%%%%%%%%%%%%%%%%%%%%%
% 講演者についての情報
\PresenterInfo
%%%%%%%%%%%%%%%%%%%%%%%%%%%%%%%%
% 講演数(半角数字)
{1}
%%%%%%%%%%%%%%%%%%%%%%%%%%%%%%%%
% 氏名
{谷口暁星}
%%%%%%%%%%%%%%%%%%%%%%%%%%%%%%%%
% 氏(ひらがな, 氏名が英字の場合はalphabet)
{たにぐち}
%%%%%%%%%%%%%%%%%%%%%%%%%%%%%%%%
% 名(ひらがな, 氏名が英字の場合はalphabet)
{あきお}
%%%%%%%%%%%%%%%%%%%%%%%%%%%%%%%%
% 所属機関(機関名(◯◯大学、◯◯研究所、など)のみ)
{名古屋大学}
%%%%%%%%%%%%%%%%%%%%%%%%%%%%%%%%
% 会員種別(半角英小文字)
%   a=正会員(一般)
%   b=正会員(学生)
%   c=準会員(一般)
%   d=準会員(学生)
%   e=非会員(一般)〔企画セッションのみ〕
%   f=非会員(学生)〔企画セッションのみ〕
{a}
%%%%%%%%%%%%%%%%%%%%%%%%%%%%%%%%
% 会員番号(半角数字4桁)
%   入会申請中の場合、受付番号(半角英大文字+半角数字5桁)
{5892}
%%%%%%%%%%%%%%%%%%%%%%%%%%%%%%%%
% メールアドレス(半角)
{taniguchi@a.phys.nagoya-u.ac.jp}
%%%%%%%%%%%%%%%%%%%%%%%%%%%%%%%%%%%%%%%%%%%%%%%%%%%%%%%%%%%%%%%%
% 講演についての情報
\PaperInfo
%%%%%%%%%%%%%%%%%%%%%%%%%%%%%%%%
% 記者発表(半角英小文字)
%   申請する場合のみ「y」を記入
{}
%%%%%%%%%%%%%%%%%%%%%%%%%%%%%%%%
% 講演分野(半角)
%  [通常セッション]
%   M=太陽
%   N=恒星・恒星進化
%   P1=星・惑星形成(星形成)
%   P2=星・惑星形成(原始惑星系円盤)
%   P3=星・惑星形成(惑星系)
%   Q=星間現象
%   R=銀河
%   S=活動銀河核
%   T=銀河団
%   U=宇宙論
%   V1=観測機器(電波)
%   V2=観測機器(光赤外・重力波・その他)
%   V3=観測機器(X線・γ線)
%   W=コンパクト天体
%   X=銀河形成・進化
%   Y=天文教育・広報普及・その他
%  [企画セッション]
%   Z1=天文データ科学の新しい潮流
%   Z2=次世代 Very Large Array (ngVLA) で
%      切り拓く新しい天文学の地平
%   Z3=富岳時代のシミュレーション天文学
{Z1}
%%%%%%%%%%%%%%%%%%%%%%%%%%%%%%%%
% 講演形式(半角英小文字)
%   a=口頭講演
%   b=ポスター講演(口頭有)
{a}
%%%%%%%%%%%%%%%%%%%%%%%%%%%%%%%%
% キーワード(5つまで)
%   分野Y以外は PASJ keyword list から選択
{methods: data analysis}
{methods: observational}
{methods: statistical}
{}
{}
%%%%%%%%%%%%%%%%%%%%%%%%%%%%%%%%
% 題名
{次世代大型サブミリ波望遠鏡の分光観測に向けたデータ科学の応用}
%%%%%%%%%%%%%%%%%%%%%%%%%%%%%%%%
% 氏名及び所属(複数の場合は「, 」で区切)
{
    谷口暁星,
    田村陽一 (名古屋大学),
    遠藤光,
    Stefanie Brackenhoff (TU Delft),
    唐津謙一 (SRON),
    竹腰達哉 (北見工業大学),
    河野孝太郎 (東京大学),
    川邊良平 (国立天文台),
    池田思朗 (統計数理研究所)
}
%%%%%%%%%%%%%%%%%%%%%%%%%%%%%%%%%%%%%%%%%%%%%%%%%%%%%%%%%%%%%%%%
\begin{document}
%%%%%%%%%%%%%%%%%%%%%%%%%%%%%%%%%%%%%%%%%%%%%%%%%%%%%%%%%%%%%%%%
% 本文開始
%%%%%%%%%%%%%%%%%%%%%%%%%%%%%%%%%%%%%%%%%%%%%%%%%%%%%%%%%%%%%%%%
サブミリ波帯での大口径($\sim$50~m)単一望遠鏡は、宇宙の3次元的体積、すなわち広視野・広波長域を一挙に分光撮像可能な望遠鏡として高い注目を集めており、建設計画は世界的な潮流へと発展している(e.g., LST, 河野他2020年秋季年会; AtLAST)。
現在、科学的要求に応えるための主力装置として、数十から数百~GHzの同時分光を実現する要素技術が開発されつつある(e.g., DESHIMA, Endo et al. 2019a/b; FINER, 田村他2020年秋季年会)。
一方、分光点数や時間サンプリング点数が桁違いに大きくなるため、データ科学の方法論を観測へ応用することが急務である。
実際、地上の望遠鏡観測で必須となる地球大気放射の除去は、多項式フィットによる既存の方法では広帯域観測への適用が難しく、観測感度を制限する原因となる。
また、テラバイト級の観測データの扱いも大きな課題である。

本講演では、最新装置開発におけるデータ科学の応用例を紹介する。
分光観測データは時間$\times$周波数の行列で表されるため、行列を低ランクな背景成分とそれ以外の前景成分に分離する統計的手法群と相性が良い。
この際、前景成分、すなわち天体信号の持つ統計的性質に即した手法を選択すれば、様々な観測へ応用できる。
従来の分光帯域($\sim$数~GHz)では、観測感度を向上しつつ大気放射を除去を達成した(Taniguchi et al. 2020)。
広帯域では、天体信号が観測帯域に対してスパースである性質や、大気特有のスペクトル形状が利用できる。
DESHIMAの広帯域シミュレーションに対する統計的手法の適用も紹介し、数年以内に実現する広帯域観測に向けた展望と課題を述べる。
%%%%%%%%%%%%%%%%%%%%%%%%%%%%%%%%%%%%%%%%%%%%%%%%%%%%%%%%%%%%%%%%
% 本文終了
%%%%%%%%%%%%%%%%%%%%%%%%%%%%%%%%%%%%%%%%%%%%%%%%%%%%%%%%%%%%%%%%
\end{document}
